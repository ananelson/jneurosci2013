\documentclass[11pt]{article}
% Change "article" to "report" to get rid of page number on title page
\usepackage{amsmath,amsfonts,amsthm,amssymb}
\usepackage{setspace}
\usepackage{fancyhdr}
\usepackage{lastpage}
\usepackage{extramarks}
\usepackage{chngpage}
\usepackage{soul}
\usepackage[round]{natbib}
\usepackage[usenames,dvipsnames]{color}
\usepackage{graphicx,float,wrapfig}
\usepackage{listings}
\usepackage{framed}
\usepackage{inconsolata}
\usepackage{hyperref}

\hypersetup{
  colorlinks = true, %Colours links instead of ugly boxes
  urlcolor = blue, %Colour for external hyperlinks
  linkcolor = blue, %Colour of internal links
  citecolor = red %Colour of citations
}

% Stuff to change, you know, if you want.
\setlength{\parindent}{1cm}
\setlength{\parskip}{3pt}

% In case you need to adjust margins:
\topmargin=-0.45in      %
\evensidemargin=0in     %
\oddsidemargin=0in      %
\textwidth=6.5in        %
\textheight=9.0in       %
\headsep=0.25in         %

\pagestyle{empty}

\newcommand{\scalefigone}[3]{
  \begin{figure}[ht!]
    % Requires \usepackage{graphicx}
    \centering
    \includegraphics[width=#2\columnwidth]{#1}
    \caption{#3}
    \label{#1}
  \end{figure}}

\setlength\fboxrule{1pt}

\graphicspath{{../figures/}}

%%%%%%%%%%%%%%%%%%%%%%%%%%%%%%%%%%%%%%%%%%%%%%%%%%%%%%%%%%%%%
% Make title
\title{Adaptive control of action timing with a spiking neural integrator
  model of anterior cingulate cortex}
\date{}
\author{Trevor Bekolay, Chris Eliasmith, Mark Laubach}
%%%%%%%%%%%%%%%%%%%%%%%%%%%%%%%%%%%%%%%%%%%%%%%%%%%%%%%%%%%%%
\begin{document}

\maketitle

\begin{abstract}
  Animals can learn to perform a simple reaction time task
  with a predictable foreperiod
  by predicting the time of the cue,
  achieving reaction times faster
  than through cue-responding alone.
  The outcome of the previous trial
  is one factor in determining
  whether an animal will attempt to time
  the foreperiod.
  The anterior cingulate cortex (ACC) has been implicated
  in enabling control of action timing
  depending on the outcome of the previous trial.
  Analysis of ACC activity suggests that
  neural integration is a key mechanism
  for adaptive control in precisely timed tasks.
  We show that a spiking neural circuit
  consisting of coupled neural integrators
  captures the neural dynamics
  of the experimentally recorded ACC.
  We use this coupled integrator model
  to adaptively control
  a spiking neural circuit that follows
  a cue-response strategy,
  and show that it performs the simple reaction time task
  with similar reaction times as experimental subjects.
  We then alter the model
  in a way consistent with
  how the aged brain degrades,
  and predict that aged subjects
  will adopt a cue-response strategy.
\end{abstract}

\section{Introduction}

The anterior cingulate cortex has been implicated
in a wide variety of functions,
including conflict monitoring
\citep{Aarts2009, Botvinick2004, VanVeen2004},
error monitoring \citep{Brown2005, Carter1998, Fecteau2003,
  Modirrousta2008, Rushworth2004},
anticipatory control \citep{Aarts2008, Koyama2001},
arousal of the sympathetic nervous system
\citep{Critchley2003, Luu2003},
and control of action timing \citep{Muir1996, Mulert2003,
  Naito2013, Narayanan2009, Risterucci2003}.
Many have proposed integrative or general models
that would allow the ACC to be involved
in several of these functions,
depending on the behavioral context
\citep{Botvinick2004, Luu2003}.
However, while there have been
attempts at modeling the ACC at the level
of artificial neural networks
\citep{Botvinick2004, Brown2005},
there are currently no mathematical
accounts of ACC function that would
explain the myriad of behaviors
modulated by the ACC.
Similarly, no existing models
have been implemented with spiking neurons,
and are therefore not directly comparable
to neural activity.

In this paper, we propose that a
two-dimensional dynamical system
can capture the activity of the ACC.
We test the model in a simple reaction time task
with a predictable foreperiod and
show that the mathematical model
enables precise timing of actions,
and changes behavior depending
on the outcome of the last trial.
We implement the dynamical system
in a spiking neural network
based on \citet{Singh2006}
and show that simulated neural activity
closely matches recorded ACC activity.
We show that the model
can be combined with
a motor control model,
resulting in behavioral changes
analogous to previous behavioral theories
of ACC function.
Finally, we damage the model in a way analogous
to aging, and predict that
aged subjects will adopt a more conservative
strategy due to a reduction of the
ability of the ACC to control action timing.

Our mathematical account of ACC function
is applied directly to
error monitoring and
the control of action timing.
However, we assert that the dynamical systems
approach is general and can be used
to explain other theories
of high-level ACC function.

\section{Simple reaction time task} \label{sec:simple-rt}

The simple reaction time task
serves as a minimal task in which
to examine the control of precisely timed actions.
In this task, the subject presses and holds down a lever
for a fixed amount of time (the ``foreperiod'';
1 second for this study).
After the foreperiod, a cue is presented
(e.g., an auditory tone).
The subject then has 0.6 seconds (the response window)
in which to release the lever.
If the lever is released in that time,
the trial is a success, and the subject is rewarded.
If the lever is released during the foreperiod,
the trial is classified as a ``premature error,''
and is penalized by a time-out.
If the lever is not released within the response window,
the trail is classified as a ``late error,''
and is also penalized by a time-out.
Figure~\ref{fig1} presents a schematic view of the task.

\section{Analysis of experimental data}

\citet{Narayanan2009} recorded
the anterior cingulate cortex (ACC) and motor cortex
of rodents performing this simple reaction-time task.
They used principal component analysis
on normalized spike densities
in order to determine
what is being represented in these brain areas
during this task.
As shown in Figure~\ref{fig2},
the first two principal components
found from analyzing the ACC data are large,
and are hypothesized to track
1) if the subject was performing the task, and
2) the relative amount of time the subject has been
performing the task.
\citeauthor{Narayanan2009} also found that the neural activity
in the ACC changed significantly depending on
the outcome of the previous trial;
the first two principal components
change significantly after errors,
and appear to contain information that an error occurred.
This information would be necessary
in order to adapt behavior based on the last trial.

Interestingly, in the post-correct case,
the first principal component
is highly correlated with the integral
of the second principal component,
and vice versa.
In the post-error case,
these correlations are significantly weaker.
This result suggests that
the ACC is performing some kind of integration..

\section{Double integrator model}

One model of neural integration
was proposed in the context of working memory
by \citet{Singh2006}.
Their model, called the \textit{double integrator} model,
uses multidimensional neurons
in integrative populations
to exhibit time-varying activity
during a vibrotactile discrimination task.
It relies on multiple coupled integrators,
whose activity resembles
the two strong principal components
in the ACC data.
The model employs
the Neural Engineering Framework (NEF; \citealp{Eliasmith2003})
to implement a spiking neural network
that captures the dynamics of the
integrative dynamical system
(see section \ref{sec:nef}).
We hypothesize that the \citeauthor{Singh2006} model
will exhibit the same neural dynamics
as the experimental ACC in the simple RT task.

\subsection{Dynamics}

The dynamics of the double integrator model
are captured in the following dynamics equations.
\begin{align*}
  \dot{x}_1 &= u \\
  \dot{x}_2 &= \alpha x_1,
\end{align*}
where $u$ represents an input signal
and $\alpha$ is the strength of
the connection between the two integrators.
This results in a two-dimensional system
in which $x_1$ integrates its input,
maintaining the accumulated value over time.
$x_2$ integrates the value of $x_1$.
\citet{Singh2006} have proposed that this system
enables time tracking;
because $x_2$ integrates $x_1$,
the amount of time elapsed since
$x_1$ changes can be determined
based on how much $x_2$ has changed.

In the simple reaction time task,
the input, $u$, is the activation of neurons
that effect the lever press.
The system is also modified by signals
representing the outcome of a particular trial;
on a correct trial, reward delivery ($R > 0$)
resets the system to a starting state.
On an error trail, a second input signal ($E < 0$)
drives $x_1$ to a low state.
When implemented in a spiking network,
this causes $x_2$ to saturate at a low value,
effectively breaking the integration,
as is seen in Figure~\ref{fig3} in post-error trails.
This results in the following dynamics.
\begin{align} \label{dint}
  \dot{x}_1 &= u + E - R x_1 \nonumber \\
  \dot{x}_2 &= \alpha x_1 - R x_2.
\end{align}
See Figure~\ref{fig4} for a graphical
depiction of the double integrator model.

In addition to the depicted system,
we inject a slowly oscillating signal into $x_1$.
This is motivated by the observation
that system \eqref{dint} is stationary
during the intertrial interval;
$x_1$ is either near 0 on rewarded trials
or some low value on error trials.
$x_2$ does not change on rewarded trials,
and saturates at some low value on error trials.
Figure~\ref{fig3} shows some activity during
the intertrial interval;
we model this additional activity
by a simple oscillation, $O = \sin(2 \pi f t + \phi)$,
where the frequency $f$ is low ($\le 0.4$Hz).
The final dynamics are
\begin{align} \label{dint-full}
  \dot{x}_1 &= u + E + O - R x_1 \nonumber \\
  \dot{x}_2 &= \alpha x_1 - R x_2.
\end{align}

Note that while there is experimental evidence for
a slow oscillations like $O$ in various brain regions
\citep{Buzsaki2004}, this injected oscillation
does not affect the performance of the network,
and is intended to model ACC activity
not being otherwise modelled in the current study.
This activity could reflect
other functions that the ACC is performing,
or could reflect the overall pace
of the simple RT task.

\section{Adaptive control circuit}

\citet{Narayanan2009} hypothesized
that the ACC activity analyzed in the present study
could be used to inhibit motor responses
until the time of the cue.
In order to verify that hypothesis,
we constructed a network that
performs the simple RT task
using the simple non-adaptive strategy
of responding to cues.
We then combined that network
with the double integrator network
such that the final model
is able to flexibly switch between
an aggressive timing strategy
and a conservative cue-response strategy
depending on the outcome of the previous trial.

\subsection{Cue-responding circuit}

The cue-responding circuit is composed
of three populations
of adaptive leaky integrate-and-fire neurons
(see Figure~\ref{fig4}).
The population labelled \textit{Trigger}
is sensitive to the cue.
The population labelled \textit{Holding}
tracks whether the lever is currently being held down.
The population labelled \textit{Press/Release}
causes lever presses and releases
through activation of different subsets
of its neurons.
The actual lever press and release
is delayed by $\sim$250ms,
the approximate amount of time
it takes for muscles to respond
to the activation of command neurons
in primary motor cortex.  % ???travis

The \textit{Holding} neurons project
to the \textit{Trigger} population
to provide the behavioral context
(i.e., \textit{can} I release the lever?)
The \textit{Trigger} neurons send projections
to the \textit{Press/Release} neurons
upon a cue that occurs in the correct context
(i.e., \textit{should} I release the lever?)

The context-driven design of this circuit
provides an alternative theory to
the prepotent inhibition model
proposed by \citet{Narayanan2009}.
They suggest that a motor command
is activated at the same time
as an inhibitory waiting signal,
and it is the removal of that inhibition
that produces the motor command.
This model proposes that
an excitatory context signal is activated,
and the pairing of that context signal
with the cue produces the motor command.
This leads to an alternative hypothesis
for the function of ACC activity:
the ACC tracks the behavioral context,
and is able to inhibit or excite motor cortex,
but does not always necessarily do so.

\subsection{Adaptive control circuit}

A weakness of the contextual cue-responding circuit
is that there is a fixed lower-bound
on the reaction time
based on the time taken to detect the cue
and for the delay between activation
of motor cortex and muscle activity.
Since the goal is to time the action
as close as possible to the time of the cue,
the cue-responding strategy may be too slow,
and so a predictive approach must be employed.

The double integrator model described
previously can implement such a strategy.
The state of the dynamical system being tracked by
the double integrator (equation \eqref{dint-full})
is approximately the same when the cue occurs.
By connecting the output of the double integrator
to the \textit{Trigger} population,
the double integrator can cause a lever release
when the state of the double integrator
is slightly before the predicted
time of the cue.

However, if the previous trial was an error,
the double integrator will not
be following the trajectory
that predicts the time of the cue,
and therefore will not cause a lever release.
The circuit then naturally
uses the conservative
cue-responding strategy.
Switching between these two strategies
depending on the outcome
of the previous trial
allows for adaptive control of action timing.

\section{Emulating aging effects}

The spiking neural implementation
of the double integrator model can be used
to emulate the effects of several pathologies.
As a proof of concept,
we emulate the effects of NMDA receptor degradation
in aging \citep{Magnusson1998}.

Feedforward connections
in the adaptive control model (Figure~\ref{fig4})
use a time constant of 10 ms,
which is consistent with the time constant
of the postsynaptic current
induced by uptake of neurotransmitter
by AMPA receptors \citep{Spruston1995}.
Recurrent connections use a time constant of 50 ms,
which is consistent with the PSC
of NMDA receptors \citep{Sah1990}.
To emulate the effects of aging on NMDA receptors,
we decrease the synaptic strengths
on these recurrent connections.

We hypothesize that this will have the same effect
as adding a small damping term, $D < 0$,
in equation \eqref{dint-full}, resulting in
\begin{align} \label{dint-damped}
  \dot{x}_1 &= u + E + O - R x_1 - D x_1 \nonumber \\
  \dot{x}_2 &= \alpha x_1 - R x_2 - D x_2.
\end{align}
This hypothesis can be tested directly
in the double integrator model,
and its effects on behavior tested
in the adaptive control circuit.

\section{Theoretical results}

\subsection{Model dynamics during the simple RT task}

The dynamical system described by equation \eqref{dint-full}
exhibits the desired behavior during simulation
of the simple RT task.
Figure~\ref{fig5} provides a breakdown of
the trajectory and vector field
throughout a correct, premature, and error trial.
Note that the oscillation is omitted
in this figure.
The final position of
the trajectory becomes the starting position
of the next trial, and therefore
provides information about
the outcome of the last trial.
It is clear from these trajectories
that the time of the cue
can be predicted,
as the ACC is only in the predictive state
when the cue occurs on correct trials.
It is also clear that the
outcome of the previous trials
is easily decoded based
on the state after the previous trial.
Note that premature trials
and late trials cannot be distinguished
on the next trial;
this is consistent with the results of
\citet{Narayanan2009}.

\subsection{The aged model loses adaptivity}

Figure~\ref{fig6} shows trajectories for
correct, premature, and late trials
for the aged model (i.e., the model with a damping term,
equation \eqref{dint-damped}).
As the damping term increases, the final position
of the trajectories on error trials
becomes closer to the final position of correct trials,
indicating that increased damping
makes distinguishing between correct and error trials
progressive more difficult.
Additionally, the distance (in state space) between
the state at which the cue occurs
and other states visited in the trajectory
decreases with increased damping,
which results in more difficulty
predicting the time of the cue
following correct trials as well.
Timing the foreperiod is therefore
a poor strategy with significant aging
(i.e., with $D \le 0.05$).

\section{Simulation results}

\subsection{Spiking networks follow predicted trajectories}

Figure~\ref{fig7} shows that
the decoded values of a spiking neural network
with the same dynamics as equation \eqref{dint-full}
also take a similar trajectory through
state space in correct, premature, and late trials.
Despite the introduction of the oscillation
and the additional noise present
in spiking neural systems,
behaviorally relevant states
are significantly separated
in the decoded state space of the spiking model.
Therefore, the cue can be accurately predicted,
and the outcome of the previous trial
has a significant effect
on the trajectory taken during a trial.

\subsection{Decreasing recurrent weights is analogous to damping}

Figure~\ref{fig8} shows that
the decoded values of a spiking neural network
in which recurrent connections are degraded
by a small amount
take similar trajectories through state space
as the damped model (Figure~\ref{fig6}).
This supports the hypothesis
that a small damping term
has a similar effect as
the degraded NMDA receptors seen in aging
in the double integrator model.

\subsection{Double integrator model has the same principal components as experimental data}

In order to perform an identical
principal component analysis
on the simulated data as was done
on the experimental data,
we randomly sampled 174 neurons
with firing rates $> 1$ Hz from
the spiking double integrator model.
The spike trains of these neurons
were convolved with a Gaussian filter ($\sigma = $ 25 ms)
and then normalized to Z-scores.
The results are shown in Figure~\ref{fig9}.
The simulated principal components
are very similar to the experimental principal components;
$R^2 = 0.96$ and $R^2 = 0.895$ for
the first two components in the post-correct case,
$R^2 = 0.979$ and $R^2 = 0.877$ for
the first two components in the post-premature case,
and $R^2 = 0.967$ and $R^2 = 0.862$ for
the first two components in the post-late case.

\subsection{Adaptive control model has similar performance as experimental subjects}

The parameters of the adaptive leaky integrate-and-fire neurons
used in all models are randomly assigned
from biologically determined distributions.
While the differences in neural parameters
has little effect on the principal components
of the double integrator model,
they can have a significant effect
on performance and reaction times
in the simple RT task.
Similarly, individual experimental subjects
have varying performance levels and reaction times
in the task.
For these reasons, we analyze
each experimental subject and instance of the model separately.

Figure~\ref{fig10} shows the performance
of the experimental subjects,
the model using only the cue-responding strategy,
and the full adaptive model.
The length of the simulations were
varied to approximately match the number of trials
as the experimental subjects.
The mean of the median reaction times
is 272$\pm$48ms for the experimental subjects,
368$\pm$34ms for the cue-responding models,
and 260$\pm$100ms for the adaptive control model.
The mean performance (\% correct trials) is
70.8\% for experimental subjects,
96.9\% for cue-responding models,
and 84.0\% for adaptive control models.
Therefore, the adaptive control model
has similar reaction times as the experimental subjects.
The cue-responding model is significantly slower,
but makes fewer errors than the adaptive control model.

\subsection{Aged model is slower and less variable}

When the aged double integrator model
is used in the adaptive control network,
the mean performance is 89.5\%,
and the mean median reaction time
is 375$\pm$47ms (see Figure~\ref{fig10} for individual results).
These results are closer
to the results of the cue-responding strategy,
suggesting that the model emulating
some effects of aging
uses the conservative cue-responding strategy
over the time prediction strategy
more often than the non-aged adaptive control network.

\section{Discussion}

The results indicate that the double integrator model
is allowing for adaptive control of action timing.
The ability to predict the time of the cue
produces faster reaction times at the cost
of additional premature errors.
The model that enables this adaptive control
has very similar neural dynamics
as the experimental ACC.

One important difference between
the experimentally recorded and simulated ACC
is in the amount of variance
explained by the principal components
and the lack of noise in the
spike density function (see Figure~\ref{fig9}).
The relative cleanness of the simulated data
is expected for several reasons.
Primarily, we expect that the ACC
is sensitive to many other factors
than are represented in the simulation.
The simulated data is more analogous
to the 174 neurons in the entire ACC
that are modulated the most during the simple RT task,
rather than the 174 neurons that happened
to be recorded during the experiments.
This could be simulated
by either injecting
noise into the simulation,
or by representing additional
randomly varying dimensions.
Additionally, the adaptive
leaky integrate-and-fire neuron model used in this study
has a more regular spiking pattern than
biological neurons,
so using a more detailed neural model
would provide a closer match
between experiment and simulation.

An interesting feature of the model
is that there is significant randomness
involved in model generation,
and no learning occurs once the model is generated.
While each instance of the model
can be thought of an individual
from the population of subjects,
one can also think of each model instance as representing
a snapshot during a subject's
learning of the task.
Therefore, the poor performance of, for example,
adaptive control model instance 10 (see Figure~\ref{fig10})
can be explained at least two ways.
It may be a subject
whose recurrent connections are too strong,
and therefore is a poor performer in the task,
or it may be a subject
who has not yet learned how to properly time
the interval, and after more trials would,
with learning, be able to make less premature errors.

The results from the aged model
form an important prediction of
how aged subjects will perform in the simple RT task.
It is notable, however, that the aged model
does not make additional premature errors;
in fact, premature errors are completely eliminated.
This is due to the targeted manner in which
aging is emulated (i.e., NMDA receptor degradation).
If the model is more generally damaged,
for example to emulate an ACC inactivation study,
we would predict that the effects
would be different than in the aging example.
In the case of overall ACC damage,
we hypothesize that the effect is more similar
to decreasing the ``resolution'' of the state space.
That is, it becomes harder to differentiate
similar states, though the system
still takes the same approximate trajectory
through the state space.
In this case, it becomes difficult
to predict the time of the cue,
and therefore additional premature errors would occur.

While we have tested our mathematical model
of ACC function in the context of
control of action timing in a simple RT task,
the tracking of behavioral variables
in a dynamical system could be used
to perform many of the other high-level functions
that the ACC is theorized to perform.
For example, in the commonly studied Stroop task,
the state space being tracked by
the ACC could be the difference between
the response suggested by word-reading and the
response suggested by the color;
the context that colors should be read
would be reflected in the passive dynamics
of the ACC, which would push the ACC
to increasingly drift towards a state
that compensates for the word-reading response.

In conclusion, we have shown that neural integration
is a key mechanism in the ACC's
ability to control the timing of actions.
Analysis of the activity of a neural integrator
model closely matches that of the experimentally recorded ACC.
That model is sufficient to allow
a simple cue-responding network
to react more quickly at the cost
of additional errors,
which matches the performance of experimental subjects.
When the model is manipulated
in a similar way as the aged brain degrades,
performance degrades and forms a prediction
of how aged experimental subjects may perform.

\bibliographystyle{plainnat}
\bibliography{jneurosci2013}

\clearpage

\appendix

\section*{Supplementary material}

\section{Methods}

\subsection{Simple reaction time task}

The simple reaction time task
(section \ref{sec:simple-rt}; Figure~\ref{fig1})
is implemented both experimentally
and in a simulated environment.

Experimentally, Long-Evans rats perform the task
in a MedPC box containing a lever,
house lights, and a spout for delivering water reward.
Rats were trained by first
associating reward with lever presses,
and then introducing progressively longer
delay periods until they reached the goal delay
of 1 second.
They were considered to have learned the task
when they performed the task correctly $>60\%$
of the time.

Theoretically, this study was performed
in the Nengo simulation environment.\footnote{Nengo is open source
  software available at \url{http://www.nengo.ca}.}
The environment was modeled using a
a finite state machine to track task state,
and scalar values between -1 and 1
to indicate the state of
house lights, lever position, auditory cue, and reward.

\subsection{Experimental data}

The experimental data analyzed in this study
is the same data presented in
\citet{Narayanan2009}.
The data consists of 174 neurons
recorded from anterior cingulate cortex with microwire arrays
in twelve male Long-Evans rats.
For more details on the data collection,
refer to the original paper.

\subsection{Simulation details}

All simulations were run
in the Nengo simulation environment
through Python scripts.
Those scripts are available
at \url{https://github.com/tbekolay}.

\subsubsection{Neural Engineering Framework} \label{sec:nef}

Nengo simplifies the construction
of spiking neural networks
using the Neural Engineering Framework (NEF; \citealp{Eliasmith2003}).
The NEF makes the assumption
that populations of spiking neurons
represent real-valued vectors,
and use a least-squares optimal method
to transform those representations
through the connections between
populations of neurons.
A population can also be recurrently connected
in order to produce dynamics,
such as the neural integrators
employed in this study.

Briefly, the representation scheme
is similar to population coding,
as proposed by \citet{Georgopoulos1986}
but extended to $n$-dimensional vector spaces.
Each neuron in a population is sensitive
to a particular direction,
called the neuron's \textit{encoder}.
The activity of a neuron can be expressed as
\begin{equation}
	a = G[\alpha \mathbf{e} \cdot \mathbf{x} + J_{bias}],
\end{equation}
where $G[\cdot]$ is the nonlinear neural activation function,
$\alpha$ is a scaling factor (gain) associated with the neuron,
$\mathbf{e}$ is the neuron's encoder,
$\mathbf{x}$ is the vector to be encoded, and
$J_{bias}$ is the background current of the cell
when $\mathbf{x} = 0$.
The currently encoded value, $\mathbf{\hat{x}}$,
can be estimated linearly.
\begin{equation}
  \mathbf{\hat{x}}(t) = \sum_i \mathbf{d}_i a_i(t),
\end{equation}
where $\mathbf{d}_i$ is the decoder,
and $a_i$ is activity of neuron $i$.

Neural activity is interpreted as a
filtered spike train.
\begin{equation}
  a_i(t) = \sum_s h(t - t_s) = \sum_s e^{-(t - t_s) / \tau_{PSC}},
\end{equation}
where $h(\cdot)$ is the exponential filter
applied to each spike,
and $s$ is the set of all spikes occurring
before the current time $t$.

The decoders are found through a least-squares minimization
of the difference between the decoded estimate
and the actual encoded vector.
\begin{equation}
  \mathbf{d} = \Upsilon^{-1} \Gamma \hspace{1.8em}
  \Gamma_{ij} = \int a_i a_j dx \hspace{1.8em}
  \Upsilon_j = \int a_j \mathbf{x} dx.
\end{equation}

Connection weights computing a function
between two populations connected in a feedforward manner
$f(\mathbf{x})$ can be determined by
solving for a set of decoding weights
for that function,
\begin{equation}
  \mathbf{d}^{f(\mathbf{x})} = \Upsilon^{-1} \Gamma \hspace{1.8em}
  \Gamma_{ij} = \int a_i a_j dx \hspace{1.8em}
  \Upsilon_j = \int a_j f(\mathbf{x}) dx,
\end{equation}
and then computing the following weight matrix, $\omega$;
\begin{equation}
	\omega_{ij} = \alpha_j \mathbf{e}_j L \mathbf{d}^{f(\mathbf{x})}_i,
\end{equation}
where $i$ indexes the input population,
$j$ indexes the output population,
and $L$ is a linear operator.

Dynamical systems of the form
$\dot{\mathbf{x}} = A(\mathbf{x}) + B(\mathbf{u})$
can be created by connecting a population
to itself with a weight matrix
that computes the function $f(\mathbf{x}) = \mathbf{x}$,
setting $L = A$ on the recurrent connection,
and receiving input $\mathbf{u}$
from other neural populations
and setting $L = B$ on those connections.

\subsubsection{Double-integrator model}

In addition to the description in the main text,
it should be noted that \citeauthor{Singh2006}
discuss four possible
realizations of their double integrator model,
all of which produce the same dynamics
in the final output population.
The realizations differ in whether
the signals are internally or externally generated,
and in whether a single multidimensional integrator
or multiple one-dimensional integrators are used.

For ease of implementation,
we used multiple one-dimensional integrators
because the resulting network is more stable than
a network with a single multidimensional integrator
\citep{Singh2006}.
No microstimulation was done
during the behavioral experiments
so it is difficult to determine
if the experimental ACC
can also be thought of as
a system of coupled integrators;
preliminary simulations using
a single multidimensional integrator
produced PCs that matched no better
than the coupled integrator model,
and often worse.

The principal component analysis
shows that activity is modulated
at the time of the lever press
and when the outcome is determined;
it is likely that these signals
arrive from the motor cortex
and other cortical areas,
but we cannot rule out the possibility
that these signals are internally generated
from the presented data alone.

\subsection{Analysis}

Both experimental and theoretical experiments
produced data in the form of spike trains
and event timestamps.
These data files were read
by a customized version of Python-Neo.\footnote{Open source software
  available at \url{http://neuralensemble.org/neo/}.}
The subsequent analysis was done
with custom Python scripts.
All analysis scripts are available
at \url{https://github.com/tbekolay}.

\subsubsection{Neural analysis}

Principal component analysis (PCA)
is used to identify the important features
in spike trains around certain events.
For each neuron with an average firing rate over 1 Hz,
we isolate the spike trains from 4 seconds prior to and
4 seconds after each press event on a correct trial.
Each perievent spike train is binned in 1 ms bins and is then
convolved with a Gaussian filter ($\sigma = $ 25 ms;
PCs are consistent with many other $\sigma$ values).
The spike trains were slightly extended
in order to eliminated edge artifacts.
These spike density functions
are normalized to Z-scores,
and then averaged over all trials to
produce a matrix in which each row is
the normalized average response of a neuron
over the perievent epoch.
PCA was performed on that matrix
using singular value decomposition.
The resulting principal components
were also normalized to Z-scores.
The amount of variance accounted for
by each principal component
is computed as the square of the component's eigenvalue
over the sum of all squared eigenvalues
($s_i^2 / \sum s^2$).

\subsubsection{Behavioral analysis}

In both the experimental and theoretical studies,
the following relevant event times are recorded:
lever press, lever release, reward delivery,
trigger stimulus onset, and house light extinguishing.
Sequences of event times were used to
identify the result of each trial.
as shown in Figure~\ref{fig1}.
Filtering the set of event timestamps
for these sequences identifies each trial outcome.
Reaction times are calculated
only for correct trials,
and are defined as the difference between
trigger stimulus onset and lever release.

\section{Results}

\subsection{Adaptive control causes premature releases}

Figure~\ref{fig11} shows the decoded values
of neural populations during a trial in which
the double integrator model
causes the control model to respond faster than
the cue-responding strategy.
Figure~\ref{fig12} shows the decoded values
during a trial in which
the double integrator causes a premature response.
Late responses, as seen in Figure~\ref{fig12},
are identical with and without the double integrator model.

\clearpage

\section{Figures}

\scalefigone{fig1}{0.5}{
  The simple reaction time task used in the experiment.
  Trials are classified as correct, premature, or late
  depending on the time of lever release.
}

\scalefigone{fig2}{1.0}{
  Experimental motivation for the adaptive control model. Data from JNP 2009.
  (A) The simple reaction time task.
  (B) When lever release events occurred.
  Each trial is aligned such that the press event
  occurs at time $t=0$.
  All of the trials shown are correct trials.
  Black dots represent correct trials
  that were preceded by correct trials.
  Red dots represent correct trials
  that were preceded by error trails
  (either premature or late).
  (C) Peri-event spike rasters and histograms
  showing different neural activity
  based on the previous outcome.
  In all six plots, black represents activity
  when the last trial was correct, and red
  represents activity when the last trial was an error.
  The shaded grey areas note where
  the neural activity is significantly different
  between the two conditions (post-correct and post-error).
  In the top row, there is elevated neural activity
  before the press is initiated, following an error.
  In the middle row, there is elevated neural activity
  before the press is initiated (both)
  and during the foreperiod (right)
  following correct trials.
  In the bottom row, there is elevated neural activity
  during the whole trial (left)
  and after the trial (right)
  following error trials.
  (D) Linear regression with a GLM was performed
  on the neural activity.
  Before the press is initiated,
  18.7\% of neurons encoded the previous outcome,
  compared to less than 10\% that
  encoded the current reaction time
  and the interaction between the two.
  (E) Z-statistics showing which neurons were significantly
  modulated by the previous outcome, and when.
  (Top) Sorted by the absolute value of the Z-statistic.
  The bottom 18.7\% of the rows have high average Z-scores
  ($p < 0.05$).
  (Bottom) Sorted by the peak of the Z-statistic.
  The line is close to linear, suggesting that the
  previous outcome information is being encoded over time
  by the entire population.
}

\scalefigone{fig3}{1.0}{
  A summary of the principal component analysis done
  in JNP 2009;
  the motivation for investigating Singh \& Eliasmith 2006.
  (A) A graphical summary of principal component analysis.
  Nomarlized perievent neural data,
  in the form of the Z-scores of instantaneous firing rates,
  is organized in a matrix, with each row
  being the z-scored firing rates
  of a single neuron on a single trial.
  Singular value decomposition is performed on the matrix,
  resulting in a matrix such that
  the number of columns (time bins) is the same.
  The rows are now ordered such that the first row
  contains largest eigenvector,
  which represents the value when the original
  data is projected onto the axes of highest variance.
  (B) The fraction of variance explained
  by each singular value
  when PCA is performed on the post-correct neural data
  (top), and the post-error data (bottom).
  In both cases, the two eigenvectors with the highest
  singular values account for nearly 50\% of variance.
  The next two eigenvectors also account for an amount
  of variance higher than would be expected when
  we linearly interpolate from the smaller
  singular values.
  (C) A summary of the loadings of the two top
  eigenvector on each neuron.
  If a neuron were encoding both the first and second
  eigenvectors to the same degree,
  we would expect a horizontal line.
  The varied lines criss-crossing each other
  suggests that each individual neuron
  has different sensitivities
  to the first and second principal components.
  (D) The top three principal components
  for the post-correct trials (black)
  and the post-error trials (red).
  (E) (Top) Plotting the cumulative sum (integral)
  of the first principal component closely matches
  the second principal component, and vice-versa
  ($R^2 = 0.904$ and $R^2 = 0.939$ respectively),
  in the post-correct case.
  (Bottom) In the post-error case,
  the first two principal components are no longer
  cumulative sums of each other
  ($R^2 = 0.639$ and $R^2 = 0.676$).
  This points to neural integration
  as a potential mechanism explaining ACC activity.
  (F) Normalized spike-density functions
  for all of the neurons analyzed in the post-correct case,
  organized by the loading
  (left) on the first principal component, and
  (right) on the second principal component.
}

\scalefigone{fig4}{0.8}{
  Block diagram of the adaptive control model.
  (Left) The adaptive control model contains
  the double integrator model,
  from Singh \& Eliasmith.
  It has been modified by accepting
  both press information ($u$)
  and error information ($E$)
  to drive $x_1$, and
  reward information ($R$)
  to control the integration
  of both $x_1$ and $x_2$.
  The output of the double integrator,
  colored red, is the only component
  that interacts with the cue-responding
  circuit, by activating neurons
  in the \textit{Trigger} population.
  (Middle) The cue-responding circuit
  that is able to perform
  the simple reaction time task
  by responding to the Cue
  provided by the environment
  (see text for details).
  (Right) Values that are not modeled
  with spiking neural populations,
  and are therefore considered
  signals that are provided by the environment.
  The lever accept the decoded
  output of the \textit{Press/Release} population,
  lowering after \textit{Press} sensitive
  neurons are activated (following some delay),
  and raising after \textit{Release} sensitive
  neurons are activated (following the same delay).
}

\scalefigone{fig5}{1.0}{
  A detailed breakdown of the dynamics and
  trajectory the model takes through state space
  for (A) correct, (B) premature, and (C) late trials
  following a correct trial.
  The injected oscillation is omitted
  in these plots;
  its time dependence makes it misleading
  to draw vector fields for a particular time.
  (A) In a correct trial, the pressing of the lever
  sends the double integrator model into
  a high part of the state space.
  The connection between the two integrators
  causes the passive dynamics (panel 2)
  to have a shift towards the right.
  The cue is delivered while drifting
  to the right, and the release occurs
  near the top-right portion of the state space.
  After the successful release,
  the reward turns off both integrators,
  causing the system to return to the origin point,
  where it remains until the next trial.
  (B) In a premature trial, the lever is released
  prior to the onset of the cue (panel 2).
  Premature release causes the subject
  to detect an error, which drives
  the ACC to a low point in the state space.
  Between trials (panel 4),
  the passive dynamics of the state space
  drive the ACC to the bottom-left portion
  of the space, where it will be
  at the start of the next trial.
  (C) In a late trial, the cue occurs,
  but the lever is not released (panel 2).
  Late release causes the subject
  to detect an error, which drives the ACC
  to a low point in the state space.
  Between trials, the passive dynamics
  of the state space drive the ACC
  to the same bottom-left portion of the space
  that the premature error results in.
  Therefore, late and premature errors
  cannot be distinguished on the next trial
  in this model.
}

\scalefigone{fig6}{1.0}{
  The effects of damping on the mathematical
  model for correct (left), premature (middle),
  and late (right) trials.
  Note the different end points (marked with arrows).
  Four damping values were simulated;
  $D = [0, 0.025, 0.05, 0.075]$.
  As can be seen in the middle and right panels,
  with $D \ge 0.05$,
  correct and error trials can no longer
  be differentiated on the next trial
  because the ACC is in the same state
  regardless of the outcome of the previous trial.
}

\scalefigone{fig7}{1.0}{
  Decoded values of the double integrator model.
  They take the same trajectory through state
  space as the mathematical model (see Figure 5).
  Similar to the mathematical model,
  the ACC represents a very different state
  after correct (left) and error (middle, right)
  trials.
}

\scalefigone{fig8}{1.0}{
  The effect of aging through degrading recurrent connection weights
  on the spiking neural model for correct (left), premature (center),
  and late (right) trials.
  Note the different end points (marked with arrows).
  Weights were degraded by first solving for an appropriate
  weight matrix, and then multiplying them by four values:
  $[1.0, 0.99, 0.98, 0.97]$.
  As can be seen in the middle and right panels,
  by multiplying weights by $\le 0.98$,
  correct and error trials can no longer
  be differentiated on the next trial
  because the ACC is in the same state
  regardless of the outcome of the previous trial.
}

\scalefigone{fig9}{1.0}{
  Results of principal component analysis
  on the data generated by the double integrator model
  for correct presses (A) post-correct trails,
  (B) post-premature trials, and (C) post-late trials.
  The top panels compare the top principal components
  of the model to the experimental ACC.
  In all cases, the PCs are significantly similar
  ($R^2 > 0.86$).
  The middle panels show normalized
  peri-event spike density functions
  for 174 randomly sampled neurons from the
  simulated double integrator model.
  The bottom panels show normalized
  peri-event spike density functions
  for all 174 neurons recorded
  during the experimental study.
}

\scalefigone{fig10}{1.0}{
  Behavioural performance of the
  experiment, cue-responding, adaptive control,
  and aged adaptive control models.
  The top panels summarize the number of
  correct, premature, and late trials
  for each experimental or simulated subject.
  The number of total trials
  for the simulations was approximately
  matched to the experimental subjects.
  The bottom panels summarize the reaction times
  on correct trials for each subject.
  The dotted black line represents
  the mean of all subjects' median reaction times.
}

\scalefigone{fig11}{1.0}{
  Decoded values from a representative correct trial
  for the simple network (A) and the adaptive network (B).
  (A) In the simple network, the trigger neurons
  activate only once the cue is delivered.
  Trigger neurons activate the release neurons,
  which initiate the release of the lever.
  There is a delay ($\sim 300ms$)
  between the activation of release neurons
  and the actual release of the lever.
  (B) In the adaptive network, when
  in the appropriate state, the cue
  can be predicted by the double integrator model,
  causing the activation of trigger---and
  therefore release---neurons before
  the time of the cue. This results
  in a faster reaction to the cue.
}

\scalefigone{fig12}{1.0}{
  Decoded values from a representative late trial for the simple
  network (A), and a premature trial for
  the adaptive network (B).
  (A) Late release occurs when
  the release neurons are not sufficiently
  active in order to effect the lever release.
  Alternatively, late release can occur
  if the release signal is not faithfully
  transmitted from the release neurons
  (modeling activity in motor cortex)
  to the intermediate populations
  (modeling activity between motor cortex and muscles)
  before interfacing with the simulated environment.
  A late release in the adaptive model
  would look identical, with little activity
  in the double integrator model that predicts cues.
  (B) Premature release occurs when
  the double integrator model incorrectly predicts
  the time of the cue.
  This occurs when integration is too fast,
  and the trigger neurons are activated
  such that a lever release is effected
  before the time of the cue.
  Note that the non-adaptive cue-responding
  circuit cannot make premature releases,
  because releases are critically dependent
  on cue information.
}



\end{document}














